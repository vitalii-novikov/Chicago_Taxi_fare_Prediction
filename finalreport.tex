% Options for packages loaded elsewhere
\PassOptionsToPackage{unicode}{hyperref}
\PassOptionsToPackage{hyphens}{url}
\PassOptionsToPackage{dvipsnames,svgnames,x11names}{xcolor}
%
\documentclass[
  letterpaper,
  DIV=11,
  numbers=noendperiod]{scrartcl}

\usepackage{amsmath,amssymb}
\usepackage{iftex}
\ifPDFTeX
  \usepackage[T1]{fontenc}
  \usepackage[utf8]{inputenc}
  \usepackage{textcomp} % provide euro and other symbols
\else % if luatex or xetex
  \usepackage{unicode-math}
  \defaultfontfeatures{Scale=MatchLowercase}
  \defaultfontfeatures[\rmfamily]{Ligatures=TeX,Scale=1}
\fi
\usepackage{lmodern}
\ifPDFTeX\else  
    % xetex/luatex font selection
\fi
% Use upquote if available, for straight quotes in verbatim environments
\IfFileExists{upquote.sty}{\usepackage{upquote}}{}
\IfFileExists{microtype.sty}{% use microtype if available
  \usepackage[]{microtype}
  \UseMicrotypeSet[protrusion]{basicmath} % disable protrusion for tt fonts
}{}
\makeatletter
\@ifundefined{KOMAClassName}{% if non-KOMA class
  \IfFileExists{parskip.sty}{%
    \usepackage{parskip}
  }{% else
    \setlength{\parindent}{0pt}
    \setlength{\parskip}{6pt plus 2pt minus 1pt}}
}{% if KOMA class
  \KOMAoptions{parskip=half}}
\makeatother
\usepackage{xcolor}
\setlength{\emergencystretch}{3em} % prevent overfull lines
\setcounter{secnumdepth}{-\maxdimen} % remove section numbering
% Make \paragraph and \subparagraph free-standing
\makeatletter
\ifx\paragraph\undefined\else
  \let\oldparagraph\paragraph
  \renewcommand{\paragraph}{
    \@ifstar
      \xxxParagraphStar
      \xxxParagraphNoStar
  }
  \newcommand{\xxxParagraphStar}[1]{\oldparagraph*{#1}\mbox{}}
  \newcommand{\xxxParagraphNoStar}[1]{\oldparagraph{#1}\mbox{}}
\fi
\ifx\subparagraph\undefined\else
  \let\oldsubparagraph\subparagraph
  \renewcommand{\subparagraph}{
    \@ifstar
      \xxxSubParagraphStar
      \xxxSubParagraphNoStar
  }
  \newcommand{\xxxSubParagraphStar}[1]{\oldsubparagraph*{#1}\mbox{}}
  \newcommand{\xxxSubParagraphNoStar}[1]{\oldsubparagraph{#1}\mbox{}}
\fi
\makeatother


\providecommand{\tightlist}{%
  \setlength{\itemsep}{0pt}\setlength{\parskip}{0pt}}\usepackage{longtable,booktabs,array}
\usepackage{calc} % for calculating minipage widths
% Correct order of tables after \paragraph or \subparagraph
\usepackage{etoolbox}
\makeatletter
\patchcmd\longtable{\par}{\if@noskipsec\mbox{}\fi\par}{}{}
\makeatother
% Allow footnotes in longtable head/foot
\IfFileExists{footnotehyper.sty}{\usepackage{footnotehyper}}{\usepackage{footnote}}
\makesavenoteenv{longtable}
\usepackage{graphicx}
\makeatletter
\newsavebox\pandoc@box
\newcommand*\pandocbounded[1]{% scales image to fit in text height/width
  \sbox\pandoc@box{#1}%
  \Gscale@div\@tempa{\textheight}{\dimexpr\ht\pandoc@box+\dp\pandoc@box\relax}%
  \Gscale@div\@tempb{\linewidth}{\wd\pandoc@box}%
  \ifdim\@tempb\p@<\@tempa\p@\let\@tempa\@tempb\fi% select the smaller of both
  \ifdim\@tempa\p@<\p@\scalebox{\@tempa}{\usebox\pandoc@box}%
  \else\usebox{\pandoc@box}%
  \fi%
}
% Set default figure placement to htbp
\def\fps@figure{htbp}
\makeatother

\KOMAoption{captions}{tableheading}
\makeatletter
\@ifpackageloaded{caption}{}{\usepackage{caption}}
\AtBeginDocument{%
\ifdefined\contentsname
  \renewcommand*\contentsname{Table of contents}
\else
  \newcommand\contentsname{Table of contents}
\fi
\ifdefined\listfigurename
  \renewcommand*\listfigurename{List of Figures}
\else
  \newcommand\listfigurename{List of Figures}
\fi
\ifdefined\listtablename
  \renewcommand*\listtablename{List of Tables}
\else
  \newcommand\listtablename{List of Tables}
\fi
\ifdefined\figurename
  \renewcommand*\figurename{Figure}
\else
  \newcommand\figurename{Figure}
\fi
\ifdefined\tablename
  \renewcommand*\tablename{Table}
\else
  \newcommand\tablename{Table}
\fi
}
\@ifpackageloaded{float}{}{\usepackage{float}}
\floatstyle{ruled}
\@ifundefined{c@chapter}{\newfloat{codelisting}{h}{lop}}{\newfloat{codelisting}{h}{lop}[chapter]}
\floatname{codelisting}{Listing}
\newcommand*\listoflistings{\listof{codelisting}{List of Listings}}
\makeatother
\makeatletter
\makeatother
\makeatletter
\@ifpackageloaded{caption}{}{\usepackage{caption}}
\@ifpackageloaded{subcaption}{}{\usepackage{subcaption}}
\makeatother

\usepackage{bookmark}

\IfFileExists{xurl.sty}{\usepackage{xurl}}{} % add URL line breaks if available
\urlstyle{same} % disable monospaced font for URLs
\hypersetup{
  pdftitle={Final Report},
  pdfauthor={Novikov Vitalii, Pozdena Nicolas, Prinz Paul, Shapovalov Anton, Wadhwani Amar},
  colorlinks=true,
  linkcolor={blue},
  filecolor={Maroon},
  citecolor={Blue},
  urlcolor={Blue},
  pdfcreator={LaTeX via pandoc}}


\title{Final Report}
\author{Novikov Vitalii, Pozdena Nicolas, Prinz Paul, Shapovalov Anton,
Wadhwani Amar}
\date{}

\begin{document}
\maketitle


\newpage

\subsection{Preface}\label{preface}

This report was created as part of the Masters study program Data
Science of the University of Applied Sciences Vienna. All authors
contributed equally on this project and have been part from the
beginning and were part of all decision made.

All of the used data is openly accessible and the sources are referenced
in the chapters below.

We would like to thank Dr.~Andreas Reschreiter for his advisory role on
the report.

\newpage

\subsection{Introduction}\label{introduction}

The aim of this report was to describe and document the process of
building a fare prediction model for chicago taxi fares, based on the
openly available data from the city of chicago
(\href{https://data.cityofchicago.org/Transportation/Taxi-Trips-2013-2023-/wrvz-psew/about_data}{dataset}).
The reason for this project was to train and gain experience on machine
learning model, handling large datasets, and planning and finishing a
datadriven project from the first idea to the deployed prototype.

\subsubsection{Goals}\label{goals}

To measure the success of this project certain goals and requirements
were imposed on the project team. Some of these goals and requirements
were set by Dr.~Anderas Reschreiter as part of the assignment other
goals, the project team set for them self.

The requirements for the assignment were to develop a model that can
predict a taxi fare based on the input of two addresses (pickup and drop
off) as well as a time when the taxi should pick up a potential
customer. Therefor a shiny app should be developed that allows a user to
input these information quickly.

The goals the team set for them self are described below.

\paragraph{Qualitative Objectives}\label{qualitative-objectives}

\begin{itemize}
\item
  Develop a predictive model to estimate the fare amount for taxi trips
  in Chicago
\item
  Evaluate various modeling approaches and deploy the best solution
\end{itemize}

\paragraph{Quantitative Objectives}\label{quantitative-objectives}

\begin{itemize}
\item
  Achieve \textbf{RM (SE per 10 minutes) \textless{} \$2} on the
  validation set
\item
  Achieve \textbf{mean AE/fare \textless{} 5\%} for on the validation
  set
\item
  Compare performance across 5 different model types
\end{itemize}

\subsubsection{Models}\label{models}

The models that were considered and analysed in this report were chosen
for their effectiveness with large datasets. All models were tested and
compared with the same quantitative objectives.

\begin{itemize}
\item
  Neural network
\item
  XGBoost
\item
  Random Forest
\item
  Linear Model
\item
  GLM
\end{itemize}

\newpage

\subsection{Planned Process}\label{planned-process}

The process of this project and report can be separated in 7 steps.

\begin{enumerate}
\def\labelenumi{\arabic{enumi}.}
\item
  data aquisition
\item
  exploratory data analysis
\item
  data cleaning
\item
  dataset preparation
\item
  model training and evaluation
\item
  ui development
\item
  combining the final works
\end{enumerate}

Each of these steps is discussed in a separate chapter. During the
development and project phase multiple of those steps were done
simultaneously to enable a swift and fast process progression.

And estimated project timeline can be seen below

{[} insert timeline {]}

\newpage

\subsection{Data acquisition}\label{data-acquisition}

As mentioned in the introduction the data for this project was made
available by the city of chicago via their online data platform
\url{https://data.cityofchicago.org/}. The dataset provides information
on taxi fares from the year 2013 until the year 2023. In these ten years
212 million trips were recorded amounting to approximately nine GB of
data. The data was made available via a direct data download or an
exposed API-Endpoint.

\subsubsection{challenges}\label{challenges}

The first challenge faced in this project was the data acquisition,
although the data is openly available the downloading process showed to
be very difficult. The direct data download only allowed for all rows to
be downloaded in one file and due to the limited downloading bandwidth
from the data servers the download was not able to be completed. The API
Endpoint uses the Socrata Framework, allowing for easy querying of the
data. However the API also had an imposed limit for their bandwitdh
resulting in a slow month-wise download in 50.000 row steps.

{[}insert downloading code{]}

\subsubsection{raw data}\label{raw-data}

the data included these 23 columns.

\begin{longtable}[]{@{}llll@{}}
\toprule\noalign{}
Column Name & Description & API Field Name & Data Type \\
\midrule\noalign{}
\endhead
\bottomrule\noalign{}
\endlastfoot
& & & \\
& & & \\
& & & \\
& & & \\
& & & \\
& & & \\
& & & \\
& & & \\
& & & \\
& & & \\
& & & \\
& & & \\
& & & \\
& & & \\
& & & \\
& & & \\
& & & \\
& & & \\
& & & \\
& & & \\
& & & \\
& & & \\
& & & \\
& & & \\
\end{longtable}

\newpage

\newpage

\subsection{Data cleaning}\label{data-cleaning}

\subsubsection{data completeness}\label{data-completeness}

\subsubsection{error and impossible
measurements}\label{error-and-impossible-measurements}

\newpage

\subsection{Dataset preparation}\label{dataset-preparation}

\subsubsection{column selection}\label{column-selection}

\subsubsection{stratifying process}\label{stratifying-process}

\subsubsection{dataset sizes}\label{dataset-sizes}

\newpage

\subsection{Model training and
evaluation}\label{model-training-and-evaluation}




\end{document}
